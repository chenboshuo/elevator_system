\documentclass[../main.tex]{subfiles} % 需要pdf封面

\begin{document}

在课本\upcite{textbook}中提供的程序提供了一种键盘事件监听的解决方案,
对于每一个按键
不断监听按键状态的二进制数码
$K(t) = (k_{t-3}\,  k_{t-2}\, k_{t-1} \,k_t)_2$
其中$k_i = 1(i=t,t-1,t-2,t-3)$表示按键释放,
$k_i = 0$表示按键按下。
若$K(t) = (1111)_2$,
即四个状态检测到按键释放,
认为当前状态$S_{N(t)}=1$,
即当前为释放状态;
其中$N(t)$为$t$时刻之前状态成功更新的次数;
若$K(t) = (0000)_2$
即四次检测到按键按下,
认为当前状态$S_{N(t)} = 0$,
表示按键按下,
其余情况视为抖动。

课本程序存储中,
每个按键需要保存$S_{N(t)},S_{N(t)-1}$两个状态,
通过
\cref{alg:textbook}
判定按键按下事件。

\begin{algorithm}[H]
  \caption{课本给出的按键按下事件的判定}
  \begin{codebox}
    \Procname{is\_just\_pressed($S_{N(t)},S_{N(t)-1}$)}
      \li \If $S_{N(t)} \neq S_{N(t)-1}$ \kw{and}
        $S_{N(t)-1} \neq 0$ :
      \Then
        \li \Return \kw{True}
      \li \Else
        \li \Return \kw{False}
      \End

  \end{codebox}
  \label{alg:textbook}
\end{algorithm}

在课本给出的算法中
$S_{N(t)},S_{N(t)-1}$是两个位变量,
申请数组时无法申请位数组\upcite{sdcc},
实现时以字节为单位存储,
为了节省空间,
可以构造数码$S_t = (S_{N(t)-1}, S_{N(t)})_2$
通过\cref{alg:imporove}
判定按键事件,
这样逻辑更为简单,
给出说明之后逻辑也更直观,
且该模块的空间占用大幅度减少,
有利于程序运行。

\begin{algorithm}[H]
  \caption{优化后的按键事件判定}
  \begin{codebox}
    \Procname{is\_just\_pressed($S_t$)}
      \li \If $S_t = (10)_2$ :
      \Then
        \li \Return \kw{True}
      \li \Else:
        \li \Return \kw{False}
      \End
  \end{codebox}
  \label{alg:imporove}
\end{algorithm}

\end{document}
