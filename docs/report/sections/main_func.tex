\documentclass[../main.tex]{subfiles} % 需要pdf封面

\begin{document}

本项目是电梯系统,
主体程序是对电梯运行的模拟,
主程序运行逻辑大致如下:

% \cref{alg:}
\begin{algorithm}[H]
  \caption{主程序}
  \begin{codebox}
    \Procname{main()}
      \li \While 电梯系统启动:
      \Then
        \li 检查请求 \label{ln:requests}
        \li 确定电梯运行方向 \label{ln:direction}
          \Comment 方向分为"上行","下行","静止"
        \li \If 电梯运行方向不为"静止"
        \Then
          \li 电梯运行到对应楼层
        \End
        \li \If 该楼层有请求:
        \Then
          \li 关闭请求
          \li 开门 \label{ln:open_door}
          \li 载客等待
          \li 关门 \label{ln:close_door}
        \End
      \End
  \end{codebox}
  \label{alg:main}
\end{algorithm}

接下来对这个大致的流程做具体分析
第\ref{ln:requests}行需要系统检查系统,
我们称为请求系统,
它需要检查电梯内有无目标楼层请求和电梯外呼叫请求;
第\ref{ln:direction}行需要一个判定移动方向的机制,
我们称为方向判定系统,
它属于调度系统的一部分,
调度系统还应该包括第\ref{ln:open_door}到\ref{ln:close_door}
使用的开关门控制系统,
他们决定要在楼层停留多长时间。
此外还有显示系统来显示各个系统的状态。

实现这些系统需要对应的算法和数据结构,
本文利用计算机常用的分层思想进行分析实现,
最终系统大致结构如%
\cref{fig:structure}

\begin{figure}[H]
  \centering
  \def\svgwidth{\linewidth}
  \input{figures/structure.pdf_tex}
  \caption{电梯系统的架构图}
  \label{fig:structure}
\end{figure}



\end{document}
