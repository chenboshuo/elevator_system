\documentclass[../main.tex]{subfiles} % 需要pdf封面

\begin{document}

在本项目中,
我尝试了大量的封装工作,
这可能是很多硬件程序员不做的事情,
因为他们对硬件的足够熟悉,
封装会降低性能,
但是因为我硬件指示的匮乏,
需要更好的展示自己程序的逻辑来减少错误,
所以需要对运行逻辑有更深层次理解,
有时候需要思考在中断计时期间能否完成这些操作,
在单片机环境中更加体现了代码可读性和程序运行效率的权衡

在对单片机的软硬件学习中,
对计算机的系统有更深层次的理解,
单片机让我们实现了一些上层接触不到的东西,
也是和物理原理呼应最完美的一层,
是了解其他工科的窗口。

从程序设计来看,
似乎可以看到为什么一些程序员讨厌面相对象的语言,
因为在C语言中可以完成一些对象化的操作,
大多数情况下,
一些新建对象的操作只是为了封装而已,
在C语言有static 关键字可以完成部分降低变量作用范围的功能,
而在单片机的程序设计中体现的尤其明显。

\end{document}
