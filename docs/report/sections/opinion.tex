\documentclass[../main.tex]{subfiles} % 需要pdf封面

\begin{document}

对于电梯系统,
是生活中常见的嵌入式系统的应用领域,
本项目的模拟也是十分简单与粗糙,
电梯的很多机制没有被模拟表现,
如电梯在开门之前要通过传感器检测是否到达楼层
\upcite{patjoshi2010design}
也没有模拟电机的运行,
但是在这次模拟中大致展示了运行逻辑和调度算法,
很多在电梯系统中讨论的问题也为其他调度优化的工程提供了新的关注点,
如电梯系统中并不是一味地追求效率,
而是响应请求,防止饥饿
\upcite{alg}.
同时本程序模拟两个电梯机位,
如果电梯数量更多,
可能考虑多处理机的通信问题,
会进一步提高系统运行效率和可靠性。


在本项目中,
我尝试了大量的封装工作,
这可能是很多硬件程序员不做的事情,
因为他们对硬件的足够熟悉,
封装会降低性能,
但是因为我硬件指示的匮乏,
需要更好的展示自己程序的逻辑来减少错误,
所以需要对运行逻辑有更深层次理解,
有时候需要思考在中断计时期间能否完成这些操作,
在单片机环境中更加体现了代码可读性和程序运行效率的权衡。
在最终实现显示效果不是太好,
由于中断函数运行效率不高的原因,
显示有些抖动,
需要进一步优化程序提高效率。
随着系统扩充,
需要对数据结构做一些调整。
在系统分析上,
在设计之前使用数学语言思考丽清逻辑十分重要,
在实现之前初步构思越严谨后期的错误就越少,
碰到错误不只是修复了代码,
还要修复逻辑与思维的漏洞。

在对单片机的软硬件学习中,
对计算机的系统有更深层次的理解,
单片机让我们实现了一些上层接触不到的东西,
也是和物理原理呼应最完美的一层,
是了解其他工科的窗口。
同时了解到看似简单的调度系统有一些复杂的逻辑,
强化了计算机专业应有的观察逻辑,流程的意思。


从程序设计来看,
似乎可以看到为什么一些程序员讨厌面相对象的语言,
因为在C语言中可以完成一些对象化的操作,
大多数情况下,
一些新建对象的操作只是为了封装而已,
在C语言有static 关键字可以完成部分降低变量作用范围的功能,
而在单片机的程序设计中体现的尤其明显。

\end{document}
