\documentclass[../main.tex]{subfiles} % 需要pdf封面

\begin{document}
简单的LED的直接显示数据即可,
这里我们讨论
LED 在按键按下时的闪烁过程闪烁,
闪烁的本质是亮暗交替,
随着$t$变化来修改数据,
最后统一刷新。

我们设控制某一LED灯
的时钟数码是$c(t)$,
每次更新的时候
$c(t) \gets \lfloor c(t)/2 \rfloor = c(t) \text{>>} 1$
也就是每次右移一次,
$c(t) = c_0 = 0$时不做任何操作,
这样的好处是$c(t)=0$之后可以统一右移更新,
不会影响数值,
如果记录$t$,
更新时$t \gets t+1$
有可能发生溢出,
会使计时器混乱,
% 即使忽略溢出的情况,
为了不发生溢出,
该时钟需要更多的存储空间(int),
也需要额外判断$t$是否在给定范围在决定是否更新,
带来额外开销,
而这样计时只需要一个时钟来决定移位的周期,
每个键盘的存储空间只要8 bit(unsigned int),
这种计时方案只要理解$c(t)$的含义逻辑就很简单,
$c(t)$本质上用其数码的二进制位数计时,
此外,
如果请求被关闭,
只需要将对应的$c(t)$清零,
如果用$t$计时,
可能需要额外数据结构记录按键是否可以继续进行闪烁操作。

具体计时方案如%
\cref{fig:blink},
其中
\[
  \begin{matrix}
    c_0 = (0000 \; 0000)2 \\
    c_1 = (1111\; 1111)_2 &
    c_2 = (0111\; 1111)_2 &
    c_3 = (0011\;1111)_2 &
    c_4 = (0001\;1111)_2 \\
    c_5 = (0000\;1111)_2 &
    c_6 = (0000\;0111)_2 &
    c_7 = (0000\;0011)_2 &
    c_8 = (0000\;0001)_2 &
  \end{matrix}
\]

\begin{figure}[H]
  \centering
  \def\svgwidth{\linewidth}
  \input{figures/blink.pdf_tex}
  \caption{闪烁小灯的计方案}
  \label{fig:blink}
\end{figure}

\end{document}
