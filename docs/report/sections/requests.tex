\documentclass[../main.tex]{subfiles} % 需要pdf封面

\begin{document}

调度系统的核心是数据结构,
用响应数据结构表示请求信息。
接下来我们构造数据结构来表示请求

设运行的方向集合$\gls{direction} = \{-1,0,1\}$,
其中用$1$标记上行,
$0$标记静止,
$-1$标记下行,
$d(t) \in D$为电梯在时刻$t$的运行方向。
$f(t)$为电梯在$t$时刻的楼层。
电梯内请求集合记录了目标楼层,
记为
\begin{equation}
  R = \{f(s) | f(s) \text{为目标楼层},s \leq t \}
  \label{eq:requests}
\end{equation}
使用(\ref{eq:requests})的数据结构定义,
楼层请求$f(t)$到达,只需要
$R \gets R \cup \{f(t)\}$

电梯外的呼叫请求需要记录发送呼叫请求的楼层和预期运行方向,
即呼叫请求集合
\begin{equation}
  C = \left\{ (f(s),D_{f(s)}) |
  f(s) \text{为呼叫请求的来源楼层}, s \leq t,
  D_{f(s)} \text{为请求方向的集合}, D_{f(s)} \subset \{-1,1\}
  \right\}
  \label{eq:call}
\end{equation}
使用(\ref{eq:requests}),(\ref{eq:call})的数据结构定义,
对$N$层楼的电梯系统,
初始状态时呼叫集合为$C=\{(i,D_i)|i=1,2,\dots,N,D_i = \emptyset\}$
请求$f(t),d(t)$到达,
只需要$D_{f(t)} \gets D_{f(t)} \cup \{d(t)\}$

以上数据结构是集合,
常用操作是并集和判断元素是否在目标集合中,
所以集合我们用数组直接实现

\begin{lstlisting}[language=c]
/// 电梯内的请求
unsigned char has_requested[2][3] = {{FALSE, FALSE, FALSE},
                                     {FALSE, FALSE, FALSE}};

/// 楼层请求
unsigned char has_called[2][3] = {{FALSE, FALSE, FALSE}, {FALSE, FALSE, FALSE}};
\end{lstlisting}


\end{document}
