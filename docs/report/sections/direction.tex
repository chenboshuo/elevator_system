\documentclass[../main.tex]{subfiles} % 需要pdf封面

\begin{document}
调度系统中需要决定运行方向,
大致原理如%
\cref{fig:direction}

\begin{figure}[H]
  \centering
  \def\svgwidth{0.6\linewidth}
  \documentclass[../main.tex]{subfiles} % 需要pdf封面

\begin{document}
调度系统中需要决定运行方向,
大致原理如%
\cref{fig:direction}

\begin{figure}[H]
  \centering
  \def\svgwidth{0.6\linewidth}
  \documentclass[../main.tex]{subfiles} % 需要pdf封面

\begin{document}
调度系统中需要决定运行方向,
大致原理如%
\cref{fig:direction}

\begin{figure}[H]
  \centering
  \def\svgwidth{0.6\linewidth}
  \documentclass[../main.tex]{subfiles} % 需要pdf封面

\begin{document}
调度系统中需要决定运行方向,
大致原理如%
\cref{fig:direction}

\begin{figure}[H]
  \centering
  \def\svgwidth{0.6\linewidth}
  \input{figures/direction.pdf_tex}
  \caption{方向系统的简单示意图}
  \label{fig:direction}
\end{figure}

方向系统的算法如%
\cref{alg:get_direction}

\begin{algorithm}[H]
  \caption{方向确定的算法}
  \begin{codebox}
    \Procname{get\_direction($f(t),d(t),R,C$)}
      \li \If $d(t) = 0$:
        \Comment 当前无方向 \label{ln:no_direction}
      \Then
        \li \If $R \neq \emptyset $: \Comment 电梯内还有请求
        \Then
          \li $f(s) \gets f\left(\min \{s|f(s) \in R\} \right)$ \Comment 先来先服务 \label{ln:fifs}
        % \End
        \li \Else \If $C \neq \emptyset$: \Comment 电梯外有楼层呼叫请求
        % \Then
          \li $f(s) \gets
            f \left(\min \{s|(f(s),D_{f(x)}) \in C\} \right)$ \Comment 先来先服务 \label{ln:out}
        % \End
        \li \Else: \Comment 无请求就回到初始位置 \label{ln:return}
        % \Then
          \li $f(s) \gets f(0)$ \label{ln:return2}
        \End
        \li $d(t) \gets \func{sign}(f(s)-f(t))$ \Comment 到目标差的符号为电梯新的运行方向
        \li \Return $d(t)$
      \End
      \li\Else: \Comment $d(t) \neq 0$:
      \Then
        \li $R_{f(t)} \gets \{f(s) | [f(s)-f(t)]d(t) \geq 0, f(s) \in R\}$ \label{ln:req_same}
          \Comment 同方向电梯内请求集合
        \li $C_{f(t)} \gets \{f(s) | [f(s)-f(t)]d(t) \geq 0, (f(s),D_{f(s)}) \in C\}$ \label{ln:call_same}
          \zi \> \Comment 楼层在当前电梯运行方向的楼层呼叫集合
        \li \If $R_{f(t)} = \emptyset \land C_{f(t)} = \emptyset$:
          \Comment 该方向无法满足请求 \label{ln:not_fit}
        \Then
          \li \Return $0$ \Comment 清除方向 \label{ln:erase}
        \li \Else:
          \li \Return $d(t)$ \Comment 保持同方向
        \End
      \End
  \end{codebox}
  \label{alg:get_direction}
\end{algorithm}

决定电梯运行方向首先考虑最简单的情况,
当电梯当前无方向(第\ref{ln:no_direction}行)时,
检查请求并判断目标楼层,
电梯内外请求是相同的,
只需要考虑请求对应楼层(第\ref{ln:fifs},\ref{ln:out}行),
但是由于数据结构定义的不同
还是需要讨论电梯内请求还是电梯外请求,
第\ref{ln:fifs}行使用的原则是先来先服务,
尽可能的防止长时间等待,
但是在本系统的实现中,
由于内存限制,
直接寻找任意一层的请求,
复杂调度系统中应该使用这一类数据结构和调度算法。

对于无方向无请求的情况,
这里设置为返回到初始楼层
($\ref{ln:return} \sim \ref{ln:return2}$行),
这样如果楼层较高,
多个电梯可以停留在不同位置,
例如两个电梯可以一个在顶层,一个在底层
方便电梯快速响应,
当然该电梯系统只有一个电梯三层楼,
这种优化效果并不明显

如果电梯具有方向,
需要知道可以满足的请求集合
$R_{f(t)},C_{f(t)}$
(第$\ref{ln:req_same} \sim \ref{ln:call_same}$行),
如果该方向不能满足任何需要,
即
$R_{f(t)} = \emptyset \land C_{f(t)} = \emptyset$
(第\ref{ln:not_fit}行),
需要清除方向重新判定(第\ref{ln:erase}行)


具体实现的函数没有返回值,
而是直接修改电梯的信息,
这样有利于提高效率,
以便于更高效的控制。

\end{document}

  \caption{方向系统的简单示意图}
  \label{fig:direction}
\end{figure}

方向系统的算法如%
\cref{alg:get_direction}

\begin{algorithm}[H]
  \caption{方向确定的算法}
  \begin{codebox}
    \Procname{get\_direction($f(t),d(t),R,C$)}
      \li \If $d(t) = 0$:
        \Comment 当前无方向 \label{ln:no_direction}
      \Then
        \li \If $R \neq \emptyset $: \Comment 电梯内还有请求
        \Then
          \li $f(s) \gets f\left(\min \{s|f(s) \in R\} \right)$ \Comment 先来先服务 \label{ln:fifs}
        % \End
        \li \Else \If $C \neq \emptyset$: \Comment 电梯外有楼层呼叫请求
        % \Then
          \li $f(s) \gets
            f \left(\min \{s|(f(s),D_{f(x)}) \in C\} \right)$ \Comment 先来先服务 \label{ln:out}
        % \End
        \li \Else: \Comment 无请求就回到初始位置 \label{ln:return}
        % \Then
          \li $f(s) \gets f(0)$ \label{ln:return2}
        \End
        \li $d(t) \gets \func{sign}(f(s)-f(t))$ \Comment 到目标差的符号为电梯新的运行方向
        \li \Return $d(t)$
      \End
      \li\Else: \Comment $d(t) \neq 0$:
      \Then
        \li $R_{f(t)} \gets \{f(s) | [f(s)-f(t)]d(t) \geq 0, f(s) \in R\}$ \label{ln:req_same}
          \Comment 同方向电梯内请求集合
        \li $C_{f(t)} \gets \{f(s) | [f(s)-f(t)]d(t) \geq 0, (f(s),D_{f(s)}) \in C\}$ \label{ln:call_same}
          \zi \> \Comment 楼层在当前电梯运行方向的楼层呼叫集合
        \li \If $R_{f(t)} = \emptyset \land C_{f(t)} = \emptyset$:
          \Comment 该方向无法满足请求 \label{ln:not_fit}
        \Then
          \li \Return $0$ \Comment 清除方向 \label{ln:erase}
        \li \Else:
          \li \Return $d(t)$ \Comment 保持同方向
        \End
      \End
  \end{codebox}
  \label{alg:get_direction}
\end{algorithm}

决定电梯运行方向首先考虑最简单的情况,
当电梯当前无方向(第\ref{ln:no_direction}行)时,
检查请求并判断目标楼层,
电梯内外请求是相同的,
只需要考虑请求对应楼层(第\ref{ln:fifs},\ref{ln:out}行),
但是由于数据结构定义的不同
还是需要讨论电梯内请求还是电梯外请求,
第\ref{ln:fifs}行使用的原则是先来先服务,
尽可能的防止长时间等待,
但是在本系统的实现中,
由于内存限制,
直接寻找任意一层的请求,
复杂调度系统中应该使用这一类数据结构和调度算法。

对于无方向无请求的情况,
这里设置为返回到初始楼层
($\ref{ln:return} \sim \ref{ln:return2}$行),
这样如果楼层较高,
多个电梯可以停留在不同位置,
例如两个电梯可以一个在顶层,一个在底层
方便电梯快速响应,
当然该电梯系统只有一个电梯三层楼,
这种优化效果并不明显

如果电梯具有方向,
需要知道可以满足的请求集合
$R_{f(t)},C_{f(t)}$
(第$\ref{ln:req_same} \sim \ref{ln:call_same}$行),
如果该方向不能满足任何需要,
即
$R_{f(t)} = \emptyset \land C_{f(t)} = \emptyset$
(第\ref{ln:not_fit}行),
需要清除方向重新判定(第\ref{ln:erase}行)


具体实现的函数没有返回值,
而是直接修改电梯的信息,
这样有利于提高效率,
以便于更高效的控制。

\end{document}

  \caption{方向系统的简单示意图}
  \label{fig:direction}
\end{figure}

方向系统的算法如%
\cref{alg:get_direction}

\begin{algorithm}[H]
  \caption{方向确定的算法}
  \begin{codebox}
    \Procname{get\_direction($f(t),d(t),R,C$)}
      \li \If $d(t) = 0$:
        \Comment 当前无方向 \label{ln:no_direction}
      \Then
        \li \If $R \neq \emptyset $: \Comment 电梯内还有请求
        \Then
          \li $f(s) \gets f\left(\min \{s|f(s) \in R\} \right)$ \Comment 先来先服务 \label{ln:fifs}
        % \End
        \li \Else \If $C \neq \emptyset$: \Comment 电梯外有楼层呼叫请求
        % \Then
          \li $f(s) \gets
            f \left(\min \{s|(f(s),D_{f(x)}) \in C\} \right)$ \Comment 先来先服务 \label{ln:out}
        % \End
        \li \Else: \Comment 无请求就回到初始位置 \label{ln:return}
        % \Then
          \li $f(s) \gets f(0)$ \label{ln:return2}
        \End
        \li $d(t) \gets \func{sign}(f(s)-f(t))$ \Comment 到目标差的符号为电梯新的运行方向
        \li \Return $d(t)$
      \End
      \li\Else: \Comment $d(t) \neq 0$:
      \Then
        \li $R_{f(t)} \gets \{f(s) | [f(s)-f(t)]d(t) \geq 0, f(s) \in R\}$ \label{ln:req_same}
          \Comment 同方向电梯内请求集合
        \li $C_{f(t)} \gets \{f(s) | [f(s)-f(t)]d(t) \geq 0, (f(s),D_{f(s)}) \in C\}$ \label{ln:call_same}
          \zi \> \Comment 楼层在当前电梯运行方向的楼层呼叫集合
        \li \If $R_{f(t)} = \emptyset \land C_{f(t)} = \emptyset$:
          \Comment 该方向无法满足请求 \label{ln:not_fit}
        \Then
          \li \Return $0$ \Comment 清除方向 \label{ln:erase}
        \li \Else:
          \li \Return $d(t)$ \Comment 保持同方向
        \End
      \End
  \end{codebox}
  \label{alg:get_direction}
\end{algorithm}

决定电梯运行方向首先考虑最简单的情况,
当电梯当前无方向(第\ref{ln:no_direction}行)时,
检查请求并判断目标楼层,
电梯内外请求是相同的,
只需要考虑请求对应楼层(第\ref{ln:fifs},\ref{ln:out}行),
但是由于数据结构定义的不同
还是需要讨论电梯内请求还是电梯外请求,
第\ref{ln:fifs}行使用的原则是先来先服务,
尽可能的防止长时间等待,
但是在本系统的实现中,
由于内存限制,
直接寻找任意一层的请求,
复杂调度系统中应该使用这一类数据结构和调度算法。

对于无方向无请求的情况,
这里设置为返回到初始楼层
($\ref{ln:return} \sim \ref{ln:return2}$行),
这样如果楼层较高,
多个电梯可以停留在不同位置,
例如两个电梯可以一个在顶层,一个在底层
方便电梯快速响应,
当然该电梯系统只有一个电梯三层楼,
这种优化效果并不明显

如果电梯具有方向,
需要知道可以满足的请求集合
$R_{f(t)},C_{f(t)}$
(第$\ref{ln:req_same} \sim \ref{ln:call_same}$行),
如果该方向不能满足任何需要,
即
$R_{f(t)} = \emptyset \land C_{f(t)} = \emptyset$
(第\ref{ln:not_fit}行),
需要清除方向重新判定(第\ref{ln:erase}行)


具体实现的函数没有返回值,
而是直接修改电梯的信息,
这样有利于提高效率,
以便于更高效的控制。

\end{document}

  \caption{方向系统的简单示意图}
  \label{fig:direction}
\end{figure}

方向系统的算法如%
\cref{alg:get_direction}

\begin{algorithm}[H]
  \caption{方向确定的算法}
  \begin{codebox}
    \Procname{get\_direction($f(t),d(t),R,C$)}
      \li \If $d(t) = 0$:
        \Comment 当前无方向 \label{ln:no_direction}
      \Then
        \li \If $R \neq \emptyset $: \Comment 电梯内还有请求
        \Then
          \li $f(s) \gets f\left(\min \{s|f(s) \in R\} \right)$ \Comment 先来先服务 \label{ln:fifs}
        % \End
        \li \Else \If $C \neq \emptyset$: \Comment 电梯外有楼层呼叫请求
        % \Then
          \li $f(s) \gets
            f \left(\min \{s|(f(s),D_{f(x)}) \in C\} \right)$ \Comment 先来先服务 \label{ln:out}
        % \End
        \li \Else: \Comment 无请求就回到初始位置 \label{ln:return}
        % \Then
          \li $f(s) \gets f(0)$ \label{ln:return2}
        \End
        \li $d(t) \gets \func{sign}(f(s)-f(t))$ \Comment 到目标差的符号为电梯新的运行方向
        \li \Return $d(t)$
      \End
      \li\Else: \Comment $d(t) \neq 0$:
      \Then
        \li $R_{f(t)} \gets \{f(s) | [f(s)-f(t)]d(t) \geq 0, f(s) \in R\}$ \label{ln:req_same}
          \Comment 同方向电梯内请求集合
        \li $C_{f(t)} \gets \{f(s) | [f(s)-f(t)]d(t) \geq 0, (f(s),D_{f(s)}) \in C\}$ \label{ln:call_same}
          \zi \> \Comment 楼层在当前电梯运行方向的楼层呼叫集合
        \li \If $R_{f(t)} = \emptyset \land C_{f(t)} = \emptyset$:
          \Comment 该方向无法满足请求 \label{ln:not_fit}
        \Then
          \li \Return $0$ \Comment 清除方向 \label{ln:erase}
        \li \Else:
          \li \Return $d(t)$ \Comment 保持同方向
        \End
      \End
  \end{codebox}
  \label{alg:get_direction}
\end{algorithm}

决定电梯运行方向首先考虑最简单的情况,
当电梯当前无方向(第\ref{ln:no_direction}行)时,
检查请求并判断目标楼层,
电梯内外请求是相同的,
只需要考虑请求对应楼层(第\ref{ln:fifs},\ref{ln:out}行),
但是由于数据结构定义的不同
还是需要讨论电梯内请求还是电梯外请求,
第\ref{ln:fifs}行使用的原则是先来先服务,
尽可能的防止长时间等待,
但是在本系统的实现中,
由于内存限制,
直接寻找任意一层的请求,
复杂调度系统中应该使用这一类数据结构和调度算法。

对于无方向无请求的情况,
这里设置为返回到初始楼层
($\ref{ln:return} \sim \ref{ln:return2}$行),
这样如果楼层较高,
多个电梯可以停留在不同位置,
例如两个电梯可以一个在顶层,一个在底层
方便电梯快速响应,
当然该电梯系统只有一个电梯三层楼,
这种优化效果并不明显

如果电梯具有方向,
需要知道可以满足的请求集合
$R_{f(t)},C_{f(t)}$
(第$\ref{ln:req_same} \sim \ref{ln:call_same}$行),
如果该方向不能满足任何需要,
即
$R_{f(t)} = \emptyset \land C_{f(t)} = \emptyset$
(第\ref{ln:not_fit}行),
需要清除方向重新判定(第\ref{ln:erase}行)


具体实现的函数没有返回值,
而是直接修改电梯的信息,
这样有利于提高效率,
以便于更高效的控制。

\end{document}
